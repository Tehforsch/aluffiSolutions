\subsection{Naive set theory}
\excercise{1.2}{Prove that if $\sim$ is a relation on a set $S$, then the corresponding family (the set of all equivalence classes) $\mathscr{P}_{\sim} := \set{\bc{a}, a \in S}
$ is indeed a partition of $S$: that is, its elements are nonempty, disjoint and their union is $S$.}

\begin{proof}
    The elements are nonempty since
    $\forall \bc{a} \in \mathscr{P}_{\sim}$: $a \sim a \Rightarrow a \in [a]$ (reflexivity). 
    
    To prove that elements are disjoint, suppose there are two members $\bc{a}$ and $\bc{b}$ of $\mathscr{P}_{\sim}$ such that $\exists c: c \in \bc{a} \wedge c \in \bc{b}$. Then $\forall x \in \bc{a} \forall y \in \bc{b}: x \sim a \sim y$ and thus, by reflexivity, $x \sim y$. But then $x \in \bc{b}$ and $y \in \bc{a}$ and therefore $\bc{a} = \bc{b}$. 

    The union of all elements is $S$ since $\forall s \in S: s \in \bc{s} \in \mathscr{P}_{\sim}$.
\end{proof}

\excercise{1.5}{Give an example of a relation that is reflexive and symmetric, but not transitive. What happens if you attempt to use this relation to define a partition on the set? (Hint: thinking about the second question will help you answer the first one.)}

If transitivity is dropped, the subsets need not be disjoint. A simple example of a non-transitive relation on $\mathbb{Z}$ is 
\begin{align}
    a \sim b \Leftrightarrow |a - b| \leq 1.
\end{align} 
The set is then divided into the sets $\set{k, k+1}, k \in \mathbb{Z}$.

\excercise{1.6}{Define a relation $\sim$ on the set $\mathbb{R}$ of real numbers, by setting $a \sim b \Leftrightarrow b - a \in \mathbb{Z}$. Prove that this is an equivalence relation and find a compelling description for $\mathbb{R} / \sim$. }
\begin{proof}
    Reflexivity follows since $0 \in \mathbb{Z}$, symmetry and transitivity quickly follow from the closedness of $\mathbb{Z}$ under negation and addition/subtraction.
\end{proof}
$\mathbb{R} / \sim$ could (for example) be described as the interval $[0, 1)$.
